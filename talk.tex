
\documentclass{beamer}

\usepackage{comment}
\usepackage{textpos}

\mode<presentation>
{
  \usetheme{Warsaw}
  % or ...

  \setbeamercovered{transparent}
  % or whatever (possibly just delete it)
}



\usepackage[english]{babel}
% or whatever

\usepackage[latin1]{inputenc}
% or whatever

\usepackage{times}
\usepackage[T1]{fontenc}


% Delete this, if you do not want the table of contents to pop up at            
% the beginning of each subsection:                                             
\AtBeginSubsection[]
{
  \begin{frame}<beamer>{Outline}
    \tableofcontents[currentsection,currentsubsection]
  \end{frame}
}

\title{Divide and Conquer -- Quantum Style!}

\author{G.~M.~Crosswhite}

\institute{Department of Physics\\University of Washington}

\date{Thursday, March 19, 2009}

\begin{document}

\begin{frame}
  \titlepage
\end{frame}

\begin{frame}{The Grand Strategy}
    \begin{enumerate} 
        \item<1-> Divide
        \item<2-> Conquer
        \item<3-> Win
    \end{enumerate} 
\end{frame} 
\begin{frame}{Will the Dinosaur Eat the Man?}
\begin{center}
\includegraphics<1>[height=7cm]{dinodiagram-1.pdf}
\includegraphics<2>[height=7cm]{dinodiagram-2.pdf}
\includegraphics<3>[height=7cm]{dinodiagram-3.pdf}
\includegraphics<4>[height=7cm]{dinodiagram-4.pdf}
\includegraphics<5>[height=7cm]{dinodiagram-5.pdf}
\includegraphics<6>[height=7cm]{dinodiagram-6.pdf}
\includegraphics<7>[height=7cm]{dinodiagram-7.pdf}
\includegraphics<8>[height=7cm]{dinodiagram-8.pdf}
\includegraphics<9>[height=7cm]{dinodiagram-9.pdf}
\end{center}
\end{frame}
\begin{frame}{Outline}
  \tableofcontents
  % You might wish to add the option [pausesections]                            
\end{frame}

\section{Motivation}
\subsection{Classic Divide and Conquer}

\begin{frame}{A Classical System}
\begin{center}
\includegraphics<1>[height=2.333333cm]{fluiddiagram-1.pdf}
\includegraphics<2->[height=2.333333cm]{fluiddiagram-2.pdf}
\end{center}

\begin{columns}[t]
\begin{column}{4cm}
\only<3->{
Step 1 -- Divide!
$$
\begin{aligned}
A_0&=27\,\text{kL}\\
B_0&=40\,\text{kL}\\
C_0&=22\,\text{kL}\\
D_0&=16\,\text{kL}\\
\vec{S_0}&=\begin{bmatrix}A_0\,\,B_0\,\,C_0\,\,D_0\end{bmatrix}
\end{aligned}
$$
}
\end{column}
\begin{column}{3.5cm}
\only<4->{Step 2 -- Conquer!}
\only<5->{
$$
\begin{aligned}
\dot A &=H_{AB} B + H_{AD} D\\
\dot B &=H_{BA} A + H_{BC} C\\
\dot C &=H_{CB} B\\
\dot D &=H_{DA} A + H_{DB} B\\
\dot{\vec{S}}&=\textbf{H}\cdot \vec{S}
\end{aligned}
$$
}
\end{column}
\begin{column}{3cm}
\only<6->{Step 3 -- Win!}

\only<7->{$$\vec{S}(t) = \int_0^t \dot{\vec{S}}\,dt + \vec{S}_0$$}
\end{column}
\end{columns} 


\end{frame} 
\subsection{Quantum Divide and Conquer}

\begin{frame}{A Quantum System}
\begin{center}
\includegraphics<1>[height=7cm]{quantumdiagram-1.pdf}
\includegraphics<2>[height=7cm]{quantumdiagram-2.pdf}
\includegraphics<3>[height=7cm]{quantumdiagram-3.pdf}
\includegraphics<4>[height=7cm]{quantumdiagram-4.pdf}
\includegraphics<5>[height=7cm]{quantumdiagram-5.pdf}
\includegraphics<6>[height=7cm]{quantumdiagram-6.pdf}
\includegraphics<7>[height=7cm]{quantumdiagram-7.pdf}
\includegraphics<8>[height=7cm]{quantumdiagram-8.pdf}
\includegraphics<9>[height=7cm]{quantumdiagram-8-5.pdf}
\includegraphics<10>[height=7cm]{quantumdiagram-9.pdf}
\includegraphics<11>[height=7cm]{quantumdiagram-10.pdf}
\includegraphics<12>[height=7cm]{quantumdiagram-11.pdf}
\includegraphics<13>[height=7cm]{quantumdiagram-12.pdf}
\includegraphics<14>[height=7cm]{quantumdiagram-13.pdf}
\includegraphics<15>[height=7cm]{quantumdiagram-14.pdf}
\includegraphics<16>[height=7cm]{quantumdiagram-15.pdf}
\includegraphics<17>[height=7cm]{quantumdiagram-16.pdf}
\includegraphics<18>[height=7cm]{quantumdiagram-17.pdf}
\includegraphics<19>[height=7cm]{quantumdiagram-18.pdf}
\includegraphics<20>[height=7cm]{quantumdiagram-19.pdf}
\includegraphics<21>[height=7cm]{quantumdiagram-20.pdf}
\includegraphics<22>[height=7cm]{quantumdiagram-21.pdf}
\includegraphics<23>[height=7cm]{quantumdiagram-22.pdf}
\includegraphics<24>[height=7cm]{quantumdiagram-23.pdf}
\includegraphics<25>[height=7cm]{quantumdiagram-24.pdf}
\includegraphics<26>[height=7cm]{quantumdiagram-25.pdf}
\includegraphics<27>[height=7cm]{quantumdiagram-26.pdf}
\includegraphics<28>[height=7cm]{quantumdiagram-27.pdf}
\includegraphics<29>[height=7cm]{quantumdiagram-25.pdf}
\includegraphics<30>[height=7cm]{quantumdiagram-28.pdf}
\includegraphics<31>[height=7cm]{quantumdiagram-29.pdf}
\includegraphics<32>[height=7cm]{quantumdiagram-28.pdf}
\includegraphics<33>[height=7cm]{quantumdiagram-30.pdf}
\includegraphics<34>[height=7cm]{quantumdiagram-31.pdf}
\includegraphics<35>[height=7cm]{quantumdiagram-32.pdf}
\includegraphics<36>[height=7cm]{quantumdiagram-33.pdf}
\includegraphics<37>[height=7cm]{quantumdiagram-34.pdf}
\includegraphics<38>[height=7cm]{quantumdiagram-35.pdf}
\includegraphics<39>[height=7cm]{quantumdiagram-36.pdf}
\includegraphics<40>[height=7cm]{quantumdiagram-37.pdf}
\includegraphics<41>[height=7cm]{quantumdiagram-38.pdf}
\end{center}
\end{frame} 
\section{Application}

\subsection{Statics}
\begin{frame}{Crash Course in Quantum Mechanics}

\begin{enumerate}
\pause
\item The state of a quantum system is represented by a vector.
$$S^{\overbrace{\alpha\beta\gamma\delta\epsilon}^{X}}\equiv S^X \equiv \vec{S}$$
\pause
\item Time evolution of a quantum system is given by multiplication by a linear unitary operator.
$$\vec{S}(t) = \textbf{U}(t)\cdot \vec{S}(0).$$
\pause
\item This unitary can be expressed as the exponentiation of a Hermiation operator called the Hamiltonian.
$$\textbf{U}(t) = e^{i\textbf{H}t}$$
\end{enumerate}

\end{frame}


\begin{frame}{Variational Approach}

\begin{enumerate}
\pause
\item Objective:  Find the lowest value of $\omega$ such that $\textbf{H}\cdot \vec{S} = \omega \vec{S}.$
\pause
\item Equivalent to finding the state vector $\vec{S}$ to minimize the Rayleigh quotient $R(\vec{S}):=\frac{\vec{S}^*\cdot \textbf{H} \cdot \vec{S}}{\vec{S}^*\cdot \vec{S}}.$
\pause
\item Since we don't know $\vec{S}$ \emph{a priori}, make a guess with tunable parameters (an ``ansatz'') and vary these parameters to minimize $R(\vec{S}).$
\pause
\item The ``divide and conquer'' approach gives us an excellent ansatz for this purpose.
\pause
\begin{enumerate}
\item $R(\vec{S})$ can be evaluated efficiently.
\pause
\item Parameters can be optimized efficiently.
\pause
\item It does a good job of approximating the true ground state.
\end{enumerate}
\end{enumerate}

\end{frame}
\begin{frame}{Efficiently computing $\vec{S}^*\cdot\vec{S}$}
\begin{center}
\includegraphics<1>[height=7cm]{sasvector-1.pdf}
\includegraphics<2>[height=7cm]{sasvector-2.pdf}
\includegraphics<3>[height=7cm]{sasvector-3.pdf}
\includegraphics<4>[height=7cm]{sasvector-4.pdf}
\includegraphics<5>[height=7cm]{sasvector-3.pdf}
\includegraphics<6>[height=7cm]{sconjdots-1.pdf}
\includegraphics<7>[height=7cm]{sconjdots-2.pdf}
\includegraphics<8>[height=7cm]{sconjdots-3.pdf}
\includegraphics<9>[height=7cm]{sconjdots-4.pdf}
\end{center}
\end{frame}

\begin{frame}{Efficiently computing $\vec{S}^*\cdot\textbf{H}\cdot\vec{S}$}
\begin{center}
\includegraphics<1>[height=7cm]{expH-1.pdf}
\includegraphics<2>[height=7cm]{expH-2.pdf}
\includegraphics<3>[height=7cm]{expH-3.pdf}
\includegraphics<4>[height=7cm]{expH-4.pdf}
\includegraphics<5>[height=7cm]{expH-5.pdf}
\includegraphics<6>[height=7cm]{expH-6.pdf}
\includegraphics<7>[height=7cm]{expH-7.pdf}
\end{center}
\end{frame}

\begin{frame}{Efficiently varying a site}
\begin{center}
\includegraphics<1>[height=7cm]{vary-1.pdf}
\includegraphics<2>[height=7cm]{vary-2.pdf}
\includegraphics<3>[height=7cm]{vary-3.pdf}
\includegraphics<4>[height=7cm]{vary-4.pdf}
\includegraphics<5>[height=7cm]{vary-5.pdf}
\includegraphics<6>[height=7cm]{vary-6.pdf}
\includegraphics<7>[height=7cm]{vary-7.pdf}
\end{center}
\end{frame}

\begin{frame}{The Battle of Haldane-Shastry}
\includegraphics<1>[height=7cm]{proofitworks-1.pdf}
\includegraphics<2>[height=7cm]{proofitworks-2.pdf}
\includegraphics<3>[height=7cm]{proofitworks-3.pdf}
\includegraphics<4>[height=7cm]{proofitworks-4.pdf}
\includegraphics<5>[height=7cm]{proofitworks-5.pdf}
\includegraphics<6>[height=7cm]{proofitworks-6.pdf}
\includegraphics<7>[height=7cm]{proofitworks-7.pdf}
\end{frame}
\subsection{Dynamics}

\begin{frame}{Time evolution: $T=\only<1>{0}\only<2-3>{\Delta t}\only<4-8>{2\Delta t}\only<9->{4\Delta t}$}
\begin{center}
\includegraphics<1>[height=7cm]{timevolve-0.pdf}
\includegraphics<2>[height=7cm]{timevolve-1.pdf}
\includegraphics<3>[height=7cm]{timevolve-2.pdf}
\includegraphics<4>[height=7cm]{timevolve-3.pdf}
\includegraphics<5>[height=7cm]{timevolve-4.pdf}
\includegraphics<6>[height=7cm]{timevolve-5.pdf}
\includegraphics<7>[height=7cm]{timevolve-6.pdf}
\includegraphics<8>[height=7cm]{timevolve-7.pdf}
\includegraphics<9>[height=7cm]{timevolve-8.pdf}
\includegraphics<10>[height=7cm]{timevolve-9.pdf}
\end{center}
\end{frame}
\begin{frame}{Singular Value Decomposition}
\begin{center}
\includegraphics<1>[height=7cm]{SVD-1.pdf}
\includegraphics<2>[height=7cm]{SVD-2.pdf}
\includegraphics<3>[height=7cm]{SVD-3.pdf}
\includegraphics<4>[height=7cm]{SVD-4.pdf}
\includegraphics<5>[height=7cm]{SVD-5.pdf}
\includegraphics<6>[height=7cm]{SVD-5-1.pdf}
\includegraphics<7>[height=7cm]{SVD-6.pdf}
\includegraphics<8>[height=7cm]{SVD-7.pdf}
\includegraphics<9>[height=7cm]{SVD-8.pdf}
\includegraphics<10>[height=7cm]{SVD-9.pdf}
\includegraphics<11>[height=7cm]{SVD-10.pdf}
\end{center}
\end{frame}
\begin{frame}{Summary}
    \begin{itemize}
        \item Divide and Conquer is a general strategy vital to understanding complex systems.
        \item For systems with intrinsic nondeterminism, one must have a more nuanced sense of how to ``divide'' the system, but it still can be made to work.
    \end{itemize}

    References:  Search for "Crosswhite" on arxiv.org;  arXiv:cond-mat/0403313
\end{frame}
\begin{frame}{Open Questions}
    \begin{itemize}
        \item How well can these methods scale with inceased computing power?
        \item How can these ideas be made to apply to systems with more than one dimension?
    \end{itemize}
\end{frame}

\section{Future Work}

\subsection{2D Systems}


\subsection{Formal Language Theory}

\end{document}
